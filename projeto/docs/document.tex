\documentclass[
	article,
	11pt,
	twoside,
	a4paper,
	english,
	brazil,
	sumario=tradicional
	]{abntex2}

\usepackage{helvet} 
\usepackage[T1]{fontenc}		
\usepackage[utf8]{inputenc}	
\usepackage{indentfirst}
\usepackage{nomencl}
\usepackage{color}
\usepackage{graphicx}
\usepackage[export]{adjustbox}
\usepackage{microtype}
\usepackage{float}
\usepackage{pdfpages}
\usepackage{scalefnt}
\usepackage{authblk}
\usepackage{multicol}
\usepackage{amsmath}
\usepackage{wrapfig}

\linespread{1.5}

\renewcommand\Authands{ e }
\renewcommand{\familydefault}{\sfdefault}

\setlrmarginsandblock{4cm}{4cm}{*}
\setulmarginsandblock{4cm}{4cm}{*}
\checkandfixthelayout

%\usepackage[brazilian,hyperpageref]{backref}	 % 
\usepackage[alf]{abntex2cite}

\titulo{Fundamentos de contagem} 
\author[*]{Ítalo Epifânio de Lima e Silva}
\author[*]{Brunno}
\author[*]{Robson}
\affil[*]{Universidade Federal do Rio Grande do Norte, UFRN}

\local{Natal, RN}
\data{2018}

\definecolor{blue}{RGB}{41,5,195}

\makeatletter
\hypersetup{
     	%pagebackref=true,
		pdftitle={\@title}, 
		pdfauthor={\@author},
    	pdfsubject={Modelo de artigo científico com abnTeX2},
	    pdfcreator={LaTeX with abnTeX2},
		pdfkeywords={abnt}{latex}{abntex}{abntex2}{atigo científico}, 
		colorlinks=true,       		% false: boxed links; true: colored links
    	linkcolor=blue,          	% color of internal links
    	citecolor=blue,        		% color of links to bibliography
    	filecolor=magenta,      		% color of file links
		urlcolor=blue,
		bookmarksdepth=4
}
\makeatother

\makeindex


\setlrmarginsandblock{3cm}{3cm}{*}
\setulmarginsandblock{3cm}{3cm}{*}
\checkandfixthelayout

\setlength{\parindent}{1.3cm}

\setlength{\parskip}{0.2cm}  

\SingleSpacing

\begin{document}

\frenchspacing 

\maketitle

\textual

\section*{Introdução}
\addcontentsline{toc}{section}{Introdução}
\input{esqueleto/introducao.tex}

\section*{Referencial Teórico}
\addcontentsline{toc}{section}{Referencial Teórico}
\input{esqueleto/referencial.tex}

\section*{Metodologia}
\addcontentsline{toc}{section}{Metodologia}
\input{esqueleto/metodologia.tex}

\section*{Desenvolvimento}
\addcontentsline{toc}{section}{Desenvolvimento}
\input{esqueleto/desenvolvimento.tex}

\section*{Resultados e discussões}
\addcontentsline{toc}{section}{Resultados e discussões}
\input{esqueleto/resultados.tex}

\section*{Conclusão}
\addcontentsline{toc}{section}{Conclusão}
\input{esqueleto/conclusao.tex}

%Esse postextual só é usado em caso de referências

\postextual

\onecolumn{
	\bibliography{referencias/referencias.bib}
}


\end{document}